\documentclass[a4paper,12pt]{ThesisStyle}
\usepackage[utf8]{inputenc}
\usepackage{thesis-style}
\usepackage{parskip}

\begin{document}

\frontmatter

\pagenumbering{gobble}

\thispagestyle{empty}
\begin{table}[htb]
\centering
\begin{Large}
\resizebox{\textwidth}{!}{\begin{tabular}{ | l |}
  \hline
  \\
\includegraphics[scale=0.9]{imatges/logo_eps.png} \\[0.7cm]
\centerline{Projecte fi de grau}\\[1cm]
\hline
\\
Estudi: Grau en Enginyeria Informàtica\\[0.7cm]
\hline
\\
Títol: Eina de suport per a l’elaboració dels horaris dels graus de l’EPS\\[0.7cm]
\hline
\\
Document: Memòria\\[0.7cm]
\hline
\\
Alumne: Adrià Ribas Chico\\[0.7cm]
\hline
\\
Tutor 1: Dra. Marta Fort Masdevall\\
Departament: Informàtica, matemàtica aplicada i estadística\\
Àrea: Llenguatges i sistemes informàtics\\[0.7cm]
\hline
\\
Tutor 2: Dr. Antonio Rodríguez Benítez\\
Departament: Informàtica, matemàtica aplicada i estadística\\
Àrea: Llenguatges i sistemes informàtics\\[0.7cm]
\hline
\\
Convocatòria (mes/any): Juny de 2022\\[0.7cm]
\hline

\end{tabular}}
\end{Large}
\end{table}

\newpage

\begin{titlepage}

% Upper part of the page
\includegraphics[scale=0.9]{imatges/logo_eps.png} \\[1cm]
\begin{center}
\textsc{\Large Projecte Fi de Grau} \\[1cm]

% Title
\begin{spacing}{2}
\HRule \\
\textbf{\Huge Eina de suport per a l’elaboració dels horaris dels graus de l’EPS} \\
\HRule \\[0.5cm]
\end{spacing}

% Author and supervisor and other data
{
\large
\emph{Autor:} \\
Adrià \textsc{Ribas Chico} \\[1cm]
Juny 2022 \\[1cm]
Grau en Enginyeria Informàtica \\[1cm]
\emph{Tutors:} \\
Marta \textsc{Fort Masdevall} \\
Antonio \textsc{Rodríguez Benítez} \\
}

\end{center}
\end{titlepage}

\titlepage

\dominitoc

\pagenumbering{roman}

\chapter*{Resum}
\label{cap:resum}

El resum del projecte va aquí \ldots

\chapter*{Agraïments}
\label{cap:agraiments}

Per començar vull agrair molt especialment a \ldots


\tableofcontents

%\listoffigures

%\listoftables

\mainmatter

\chapter{Introducció}
\label{cap:intro}

\section{Antecedents}
\label{sec:antecedents}

En aquesta secció, es resumiran els antecedents que han donat peu al plantejament d'aquest projecte.

La idea del projecte néix de determinades necessitats que cert personal de l'\emph{Escola Politècnica Superior} de la \emph{Universitat de Girona} fa temps que
té. Més concretament, es tracta d'una necessitat del personal encarregat de gestionar tot el que fa referència a la confecció i manteniment dels horaris del
centre: horaris dels graus, dels professors, ocupació d'aules i espais, etc.

Actualment, per a dur a terme la creació dels horaris, aquestes persones utilitzen mètodes i eines poc àgils i incòmodes, a part de no estar automatitzades
ni específicament dissenyades per a abordar aquest tipus de tasques.

Ara per ara, per exemple, no tenen manera de detectar incompatibilitats horàries ni solapaments de manera automàtica. A la secció~\ref{sec:situacio_actual} es
descriuran els mètodes emprats actualment amb més profunditat.

\section{Situació actual}
\label{sec:situacio_actual}

En aquesta secció, es detallaran quins són i com funcionen els mètodes de gestió i elaboració d'horaris de l'EPS que s'utilitzen actualment. Es veuran quins càrrecs
de l'escola hi participen i les seves funcions, així com la manera en què interactuen entre ells i intercanvien la informació corresponent.

Parlar del fluxe dels processos, etc \ldots

\section{Propòsit}
\label{sec:proposit}

En aquesta secció, s'exposarà el propòsit general del projecte, tenint en compte els antecedents vists a la secció~\ref{sec:antecedents}.

En definitiva, la gestió dels horaris de l'EPS suposa una inversió de temps massa elevada per la gent que se n'ocupa. És per això, que la
\emph{Dra. Marta Fort Masdevall}, coordinadora d'estudi del Grau en Enginyeria Informàtica de la universitat, proposa un projecte de fi de grau que té la finalitat
de trobar una solució al problema descrit.

El propòsit és desenvolupar una eina de suport informàtic que permeti a l'usuari elaborar horaris de forma àgil, eficient i segura. L'eina també hauria de ser capaç
de comprovar automàticament la disponibilitat de les aules, les incompatibilitats horàries dels professors i la concordança entre assignatures i el nombre de grups
previstos. A més a més, hauria d'oferir diverses vistes per tal que l'usuari pugui visualitzar la informació, com ara:
\begin{itemize}
  \item Els horaris d'un grau per curs i quadrimestre.
  \item Els horaris d'un professor per quadrimestre.
  \item L'ocupació d'un espai per quadrimestre.
\end{itemize}

També es planteja la possibilitat de disposar d'un sistema de control d'usuaris. Cada usuari tindria assignat un conjunt de rols determinat. Els rols representarien
els diferents càrrecs del personal de l'EPS en matèria de gestió d'horaris. Així doncs, cada usuari podria executar les accions i consultar la informació que el seu
conjunt de rols li permeti. D'aquesta manera, es dividirien les diferents tasques i processos entre rols d'usuari i cadascun dels càrrecs podria realitzar la feina
que li correspon. La proposta inicial comprèn els següents rols d'usuari:
\begin{itemize}
  \item \texttt{Administrador}: Introdueix les aules i els grups previstos.
  \item \texttt{Coordinador}: Elabora els horaris.
  \item \texttt{Responsable de docència}: Dóna d'alta i assigna professors als grups.
  \item \texttt{Professor}: Visualitza el seu horari.
\end{itemize}

A més a més, l'aplicació hauria de ser accessible via \emph{web}. D'aquesta manera, tots els seus usuaris podrien utilitzar-la des de qualsevol lloc i dispositiu, sense
preocupar-se d'instal·lacions ni actualitzacions.

\section{Motivacions}
\label{sec:motivacions}

En aquesta secció, es parlarà en primera persona sobre quines motivacions personals hi ha al darrera del projecte i justifiquen la seva elecció.

\ldots

\section{Objectius generals}
\label{sec:objectius_generals}

En aquesta secció, s'enumeraran els objectius generals del projecte, que ja s'han deixat entreveure prèviament a la secció~\ref{sec:proposit}. No obstant aixó,
la llista completa és la següent:
\begin{itemize}
  \item Proporcionar una interfície còmoda i intuïtiva per a dur a terme les tasques de gestió i elaboració dels horaris de l'EPS.
  \item Emmagatzemar i processar dinàmicament les dades i relacions referents als diversos graus, cursos, quadrimestres, assignatures, grups, espais, professors, etc.
  \item Detectar i evitar automàticament qualsevol tipus d'inconsistència o incompatibilitat horària, per tal d'aportar seguretat al treball.
  \item Possibilitar la pujada d'arxius externs de dades que serveixin per obtenir les dades bàsiques necessàries pel funcionament de l'aplicació i generar possibles
  punts de partida per a la planificació dels horaris.
  \item Permetre la visualització de l'ocupació de les aules, dels horaris dels professors i dels horaris dels grups de cada grau, entre d'altres vistes que puguin ser
  d'utilitat pels usuaris.
  \item Admetre diferents rols d'usuari, als quals se'ls assigni una sèrie de tasques i un conjunt de permisos concret.
\end{itemize}

Al capítol~\ref{cap:requisits} es desenvoluparan dits objectius generals del projecte i es concretaran els requeriments específics de l'aplicatiu.


%%%%%%%%%%%%%%%%%%%%%%%%%%%%%%%%%%%%%%%%%%%%%%%%%%%%%%%%%%%%%%%%%%%%%%%%%%%%%%%%%%%%%%%%%%%%%%%%%%%%%%%%%%%%%%%%%%%%%%%%%%%%%%%%%%%%%%%%%%%%%%%%%%%%%%%%%%%%
\section{-------------- EXEMPLES I UTILITATS --------------}
\subsection{Paraules per començar seccions}
\textbf{Idees:}

Abordar, concretar, exposar, parlar, descriure, repassar, mostrar, ensenyar, desenvolupar, tractar, veure, aprofundir, investigar, discutir, indagar?, detallar,
enumerar,


\subsection{Altres}

Això és un exemple de citació d'un llibre~\cite{Coleman1974}, un article científic~\cite{Ruiz2008} i una referència a una web~\cite{Halcon}.

Exemple de taula:
\begin{table}[htb]
\centering
\begin{tabular}{ | r | c | c | l | }
 \hline
  Any & Matriculats & Aprovats & Percentatge\\
\hline
 2019  & 65 & 47 & 72.3\%\\
 2020  & 69 & 48 & 69.6\%\\
 2021  & 75 & 58 & 77.3\%\\
  \hline
  \end{tabular}
\caption{\label{taula:taulaexemple} Aquí és on s'ha de posar el peu de taula. }
\end{table}

Exemple de figura:
\begin{figure}[htb]
\centering
\includegraphics[width=8 cm]{imatges/logo_eps.png}
\caption{\label{fig:logo} Logotip de l'Escola Politècnica Superior.}
\end{figure}

Exemple de fòrmula:
\begin{equation}
H(X) = -\sum_{i=1}^{N}p_s(x_i) \log \left( p_s(x_i) \right).
\label{equ:entropia}
\end{equation}


També es pot fer referència en el text a les taules (p.ex. veure la Taula~\ref{taula:taulaexemple}), a les figures (p.ex. veure la Figura~\ref{fig:logo})
o a les fòrmules (p.ex. veure Equació~\ref{equ:entropia}).

%%%%%%%%%%%%%%%%%%%%%%%%%%%%%%%%%%%%%%%%%%%%%%%%%%%%%%%%%%%%%%%%%%%%%%%%%%%%%%%%%%%%%%%%%%%%%%%%%%%%%%%%%%%%%%%%%%%%%%%%%%%%%%%%%%%%%%%%%%%%%%%%%%%%%%%%%%%%


\chapter{Viabilitat}
\label{cap:viabilitat}



\chapter{Metodologia}
\label{cap:metodologia}



\chapter{Planificació}
\label{cap:planificacio}



\chapter{Marc de treball i conceptes previs}
\label{cap:marcdetreball}



\chapter{Requisits del sistema}
\label{cap:requisits}



\chapter{Estudi i decisions}
\label{cap:estudi}



\chapter{Anàlisi i disseny del sistema}
\label{cap:analisi}



\chapter{Implementació i proves}
\label{cap:implementacio}



\chapter{Implantació i resultats}
\label{cap:implantacio}



\chapter{Conclusions}
\label{cap:conclusions}



\chapter{Treball futur}
\label{cap:treball_futur}




\backmatter

%\appendix

%\include{Appendix1}

\bibliographystyle{ThesisStyleBreakable}
\bibliography{biblio}

%\printnomenclature

\end{document}
