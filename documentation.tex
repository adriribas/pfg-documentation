\documentclass[a4paper,12pt]{ThesisStyle}
\usepackage[utf8]{inputenc}
\usepackage{thesis-style}
\usepackage{parskip}

\begin{document}

\frontmatter

\pagenumbering{gobble}

\thispagestyle{empty}
\begin{table}[htb]
\centering
\begin{Large}
\resizebox{\textwidth}{!}{\begin{tabular}{| l |}
  \hline
  \\
\includegraphics[scale=0.9]{imatges/logo_eps.png}\\[0.7cm]
\centerline{Projecte fi de grau}\\[1cm]
\hline
\\
\textbf{Estudi}: Grau en Enginyeria Informàtica\\[0.7cm]
\hline
\\
\textbf{Títol}: Eina de suport per a l’elaboració dels horaris dels graus de l’EPS\\[0.7cm]
\hline
\\
\textbf{Document}: Memòria\\[0.7cm]
\hline
\\
\textbf{Alumne}: Adrià Ribas Chico\\[0.7cm]
\hline
\\
\textbf{Tutor 1}: Dra. Marta Fort Masdevall\\
\textbf{Departament}: Informàtica, matemàtica aplicada i estadística\\
\textbf{Àrea}: Llenguatges i sistemes informàtics\\[0.7cm]
\hline
\\
\textbf{Tutor 2}: Dr. Antonio Rodríguez Benítez\\
\textbf{Departament}: Informàtica, matemàtica aplicada i estadística\\
\textbf{Àrea}: Llenguatges i sistemes informàtics\\[0.7cm]
\hline
\\
\textbf{Convocatòria (mes/any)}: Juny de 2022\\[0.7cm]
\hline

\end{tabular}}
\end{Large}
\end{table}

\newpage

\begin{titlepage}

% Upper part of the page
\includegraphics[scale=0.9]{imatges/logo_eps.png} \\[1cm]
\begin{center}
\textsc{\Large Projecte Fi de Grau} \\[1cm]

% Title
\begin{spacing}{2}
\HRule \\
\textbf{\Huge Eina de suport per a l’elaboració dels horaris dels graus de l’EPS} \\
\HRule \\[0.5cm]
\end{spacing}

% Author and supervisor and other data
{
\large
\emph{Autor:} \\
Adrià \textsc{Ribas Chico} \\[1cm]
Juny 2022 \\[1cm]
Grau en Enginyeria Informàtica \\[1cm]
\emph{Tutors:} \\
Dra. Marta \textsc{Fort Masdevall} \\
Dr. Antonio \textsc{Rodríguez Benítez} \\
}

\end{center}
\end{titlepage}

\titlepage

\dominitoc

\pagenumbering{roman}

\chapter*{Resum}
\label{cap:resum}

El resum del projecte va aquí \ldots

\chapter*{Agraïments}
\label{cap:agraiments}

Per començar vull agrair molt especialment a \ldots


\tableofcontents

%\listoffigures

%\listoftables

\mainmatter

\chapter{Introducció}
\label{cap:intro}

\section{Antecedents}
\label{sec:antecedents}

En aquesta secció, es resumiran els antecedents que han donat peu al plantejament d'aquest projecte.

La idea del projecte neix de determinades necessitats que cert personal de l'Escola Politècnica Superior de la Universitat de Girona fa temps que
té. Més concretament, es tracta d'una necessitat del personal encarregat de gestionar tot el que fa referència a la confecció i manteniment dels horaris del
centre: horaris dels graus, dels professors, ocupació d'aules i espais, etc.

Actualment, per dur a terme la creació dels horaris, aquestes persones utilitzen mètodes i eines incòmodes i poc àgils, a part de no estar automatitzades
ni específicament dissenyades per abordar aquest tipus de tasques.

Ara per ara, per exemple, no tenen manera de detectar incompatibilitats horàries ni solapaments de manera automàtica. A la secció~\ref{sec:situacio_actual} es
descriuran els mètodes emprats actualment amb més profunditat.

\section{Situació actual}
\label{sec:situacio_actual}

En aquesta secció, es detallaran quins són i com funcionen els mètodes de gestió i elaboració d'horaris de l'EPS que s'utilitzen actualment. Es veuran quins càrrecs
de l'escola hi participen i les seves funcions, així com la manera en què interactuen entre ells i intercanvien la informació corresponent.

Parlar del fluxe dels processos, etc \ldots

Rectorat reparteix crèdits a les facultats. Cada facultat decideix quants grups grans, petits, etc. de cada grau en funció dels alumnes. Introdueix els excels.
A partir d'aquí, entren els coordinadors, per fer els horaris amb els seus mètodes. Canviar-los el mínim possible respecte l'any passat, per intentar evitar
solapaments. Un cop fets, els coordinadors passen els horaris a direcció. S'entren al sistema i revisen solapaments. Si tot està bé, els responsables de docència,
que assignen professors als grups. 

\section{Propòsit}
\label{sec:proposit}

En aquesta secció, s'exposarà el propòsit general del projecte, tenint en compte els antecedents vists a la secció~\ref{sec:antecedents}.

En definitiva, la gestió dels horaris de l'EPS suposa una inversió de temps massa elevada per a la gent que se n'ocupa. És per això que la
Dra. Marta Fort Masdevall, coordinadora d'estudi del Grau en Enginyeria Informàtica de la universitat, proposa un projecte de fi de grau que té la finalitat
de trobar una solució al problema.

El propòsit és desenvolupar una eina de suport informàtic que permeti a l'usuari elaborar horaris de forma àgil, eficient i segura. L'eina també hauria de ser capaç
de comprovar automàticament la disponibilitat de les aules, les incompatibilitats horàries dels professors i la concordança entre les assignatures i el nombre de grups
previstos. A més a més, hauria d'oferir diverses vistes per tal que l'usuari pugui visualitzar la informació, com ara:
\begin{itemize}
  \item Els horaris d'un grau per curs i quadrimestre.
  \item Els horaris d'un professor per quadrimestre.
  \item L'ocupació d'un espai per quadrimestre.
\end{itemize}

També es planteja la possibilitat de disposar d'un sistema de control d'usuaris. Cada usuari tindria assignat un conjunt de rols determinat. Els rols representarien
els diferents càrrecs del personal de l'EPS en matèria de gestió d'horaris. Així doncs, cada usuari podria executar les accions i consultar la informació que el seu
conjunt de rols li permeti. D'aquesta manera, es dividirien les diferents tasques i processos entre rols d'usuari i cadascun dels càrrecs podria realitzar la feina
que li correspon. La proposta inicial comprèn els rols d'usuari següents:
\begin{itemize}
  \item \texttt{Administrador}: Introdueix les aules i els grups previstos.
  \item \texttt{Coordinador}: Elabora els horaris.
  \item \texttt{Responsable de docència}: Dóna d'alta i assigna professors als grups.
  \item \texttt{Professor}: Visualitza el seu horari.
\end{itemize}

A més a més, l'aplicació hauria de ser accessible via web. D'aquesta manera, tots els usuaris podrien utilitzar-la des de qualsevol lloc i dispositiu, sense
preocupar-se d'instal·lacions ni actualitzacions.

\section{Motivacions}
\label{sec:motivacions}

En aquesta secció, es parlarà en primera persona sobre quines motivacions personals hi ha darrere del projecte i en justifiquen l'elecció.

\ldots

\section{Objectius generals}
\label{sec:objectius_generals}

En aquesta secció, s'enumeraran els objectius generals del projecte, que ja s'han deixat entreveure prèviament a la secció~\ref{sec:proposit}. No obstant això,
a continuació se'n presenta la llista completa:
\begin{itemize}
  \item Proporcionar una interfície còmoda i intuïtiva per dur a terme les tasques de gestió i elaboració dels horaris de l'EPS.
  \item Emmagatzemar i processar dinàmicament les dades i relacions referents als diversos graus, cursos, quadrimestres, assignatures, grups, espais, professors, etc.
  \item Detectar i evitar automàticament qualsevol tipus d'inconsistència o incompatibilitat horària, per tal d'aportar seguretat al treball.
  \item Possibilitar la pujada d'arxius externs de dades que serveixin per obtenir la informació bàsica necessària per al funcionament de l'aplicació i generar possibles
  punts de partida per a la planificació dels horaris.
  \item Permetre la visualització de l'ocupació de les aules, dels horaris dels professors i dels horaris dels grups de cada grau, entre d'altres vistes que puguin ser
  d'utilitat pels usuaris.
  \item Admetre diferents rols d'usuari, als quals s'assigni una sèrie de tasques i un conjunt de permisos concret.
\end{itemize}

Al capítol~\ref{cap:requisits} es desenvoluparan aquests objectius generals i es concretaran els requeriments específics de l'aplicatiu.


%%%%%%%%%%%%%%%%%%%%%%%%%%%%%%%%%%%%%%%%%%%%%%%%%%%%%%%%%%%%%%%%%%%%%%%%%%%%%%%%%%%%%%%%%%%%%%%%%%%%%%%%%%%%%%%%%%%%%%%%%%%%%%%%%%%%%%%%%%%%%%%%%%%%%%%%%%%%
\section{-------------- EXEMPLES I UTILITATS --------------}
\subsection{Paraules per començar seccions}
\textbf{Idees:}

Abordar, concretar, exposar, parlar, descriure, repassar, mostrar, ensenyar, desenvolupar, tractar, veure, aprofundir, investigar, discutir, indagar, detallar,
enumerar,


\subsection{Altres}

Això és un exemple de citació d'un llibre~\cite{Coleman1974}, un article científic~\cite{Ruiz2008} i una referència a una web~\cite{Halcon}.

Exemple de taula:
\begin{table}[htb]
\centering
\begin{tabular}{ | r | c | c | l | }
 \hline
  Any & Matriculats & Aprovats & Percentatge\\
\hline
 2019  & 65 & 47 & 72.3\%\\
 2020  & 69 & 48 & 69.6\%\\
 2021  & 75 & 58 & 77.3\%\\
  \hline
  \end{tabular}
\caption{\label{taula:taulaexemple} Aquí és on s'ha de posar el peu de taula. }
\end{table}

Exemple de figura:
\begin{figure}[htb]
\centering
\includegraphics[width=8 cm]{imatges/logo_eps.png}
\caption{\label{fig:logo} Logotip de l'Escola Politècnica Superior.}
\end{figure}

Exemple de fòrmula:
\begin{equation}
H(X) = -\sum_{i=1}^{N}p_s(x_i) \log \left( p_s(x_i) \right).
\label{equ:entropia}
\end{equation}


També es pot fer referència en el text a les taules (p.ex. veure la Taula~\ref{taula:taulaexemple}), a les figures (p.ex. veure la Figura~\ref{fig:logo})
o a les fòrmules (p.ex. veure Equació~\ref{equ:entropia}).

%%%%%%%%%%%%%%%%%%%%%%%%%%%%%%%%%%%%%%%%%%%%%%%%%%%%%%%%%%%%%%%%%%%%%%%%%%%%%%%%%%%%%%%%%%%%%%%%%%%%%%%%%%%%%%%%%%%%%%%%%%%%%%%%%%%%%%%%%%%%%%%%%%%%%%%%%%%%


\chapter{Viabilitat}
\label{cap:viabilitat}



\chapter{Metodologia}
\label{cap:metodologia}



\chapter{Marc de treball i conceptes previs}
\label{cap:marcdetreball}
Explicar setmanes a i b, assignatures compartides, què són els blocs (especials també)..., quantes hores
són un crèdit, ¿explicar funcions de cada rol?... 



Explicar tecnologies.

% Grups grans i mitjans no hi ha restriccions d'aules. Grups petits sí,


\chapter{Requisits del sistema}
\label{cap:requisits}

\section{Requisits generals}
\label{sec:requisits_generals} % Poder guardar versions del procés d'elaboració d'horaris.

En aquesta secció, es llistaran els requisits generals del sistema, comuns per a tots els usuaris, independentment dels seus rols.

\begin{itemize}
  \item Assegurar l'autenticació de tots els usuaris a través d'un formulari de \emph{login}. S'ha de mostrar a la pantalla quan un usuari no autenticat accedeix a l'aplicació. Sense estar-ho, no l'ha de poder fer servir.
  \item No permetre el registre d'usuaris des de fora. Usuaris de determinats rols s'encarregaran de donar d'alta altres usuaris del rol que els correspongui.
  \item Proporcionar accés als manuals d'usuari de l'aplicació, així com a la informació de contacte i als aspectes legals.
\end{itemize}

% ADMINISTRADORS
\section{Requisits dels Administradors} % TO DO
\label{sec:requisits_administradors}

En aquesta secció, es llistaran els requisits particulars dels usuaris amb rol d'Administrador.

\begin{itemize}
  \item Entra l'excel amb les previsions. Aquest també té quantes aules de cada tipus hi ha.
  \item Pot editar el nombre d'aules de cada tipus i afegir-ne, etc.
  \item Dóna d'alta noves assignatures, graus, etc (estarà a l'excel)
  \item Dóna d'alta directors de departament i coordinadors dels graus.
  \item Assigna aules als diferents blocs.
\end{itemize}

% COORDINADORS
\section{Requisits dels Coordinadors} % TO DO
\label{sec:requisits_coordinadors}

En aquesta secció, es llistaran els requisits particulars dels usuaris amb rol de Coordinador.

Bloc:

\begin{itemize} % En general fa horaris
  \item Veu horaris dels seus graus i dels graus amb assignatures compartides i ocupació d'aules.
  \item Veu horaris dels professors que fan docència a algun dels seus graus.
  \item Poder partir d'un horari d'un curs anterior.
  \item Poder esborrar els bloc que sobren.
  \item Nom
  \item Poder moure els blocs i assignar-los a franges horàries del calendari.
  \item Poder tenir blocs especials: PACs, reunions amb profes, etc.
\end{itemize}

% DIRECTORS DE DEPARTAMENT
\section{Requisits dels Director de departament} % TO DO  Els departaments estan dividits en àrees. Cada àrea té un responsable de docència.
\label{sec:requisits_director_departament}
\begin{itemize}
  \item Dóna d'alta responsables de docència.
  \item Pot veure horaris dels graus que tenen assignatures del seu departament i dels professors del departament.
\end{itemize}

% RESPONSABLES DE DOCÈNCIA
\section{Requisits dels Responsables de docència} % TO DO Està lligat a una àrea. Pot haver-hi més d'un.
\label{sec:requisits_responsables_docencia}

En aquesta secció, es llistaran els requisits particulars dels usuaris amb rol de Responsable de docència.

\begin{itemize} % Pot anar assignant quan els horaris encara s'estan fent
  \item Dóna d'alta els professors, només del seu departament.
  \item Assignen professors als grups de les assignatures de la seva àrea. (pot ser que en tinguin més d'una)
  \item Veuen horaris dels professors i les assignatures de la seva àrea.
\end{itemize}

% PROFESSORS
\section{Requisits dels Professors} % TO DO
\label{sec:requisits_professors}

En aquesta secció, es llistaran els requisits particulars dels usuaris amb rol de Professor.

\begin{itemize} % vistes per separat
  \item Veure horaris dels seus grups.
  \item Veure horaris de les assignatures on fa docència.
  \item Veure horaris dels graus en què fa docència a alguna de les seves assignatures.
\end{itemize}

\chapter{Planificació}  % Ordre canviat. Ordre antic: Planificació --> Marc de treball --> Requisits
\label{cap:planificacio}



\chapter{Estudi i decisions}
\label{cap:estudi}

\begin{itemize}
  \item Visual Studio Code. Extensions?
  \item Git i GitHub
  \item YouTrack ?
  \item Node js
  \item Express
  \item MySQL
  \item Postman
  \item \ldots
\end{itemize}

\chapter{Anàlisi i disseny del sistema}
\label{cap:analisi}



\chapter{Implementació i proves}
\label{cap:implementacio}



\chapter{Implantació i resultats}
\label{cap:implantacio}



\chapter{Conclusions}
\label{cap:conclusions}



\chapter{Treball futur}
\label{cap:treball_futur}




\backmatter

\bibliographystyle{ThesisStyleBreakable}
\bibliography{biblio}
%\printnomenclature

%\appendix

%\include{Appendix1}

\chapter*{Manual d'usuari}

\section*{Rol 1}



\section*{Rol 2}



\section*{Rol 3}



\section*{Rol 4}




\end{document}
